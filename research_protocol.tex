% !TEX encoding = UTF-8 Unicode
% !!!  THIS FILE IS UTF-8 !!!
% !!!  MAKE SURE YOUR LaTeX Editor IS CONFIGURED TO USE UTF-8 !!!
% ---------------------------------------------------------------
% Author: Adrian Joost
% ---------------------------------------------------------------

%-------------------------
% header
% ------------------------
\documentclass[a4paper,12pt]{scrartcl}
\linespread {1.25}

%-------------------------
% packages and config
% ------------------------
\input{packages_and_configuration}

%-------------------------
% document begin
%-------------------------
\begin{document}

\title{Research protocol}

\section{Introduction}

This research protocol describes how to record the search for information sources for a bachelor thesis. The preliminary research that led to the formulation of the research question should be documented under "Preparatory Research"; this may be given as a summary rather than a detailed table.

The columns of the research protocol are to be understood as follows:
\begin{description}
	\item[Date:] Date of the search.
	\item[Duration:] Time spent on the individual search (in minutes).
	\item[Search terms:] Words and combinations of words used in the search, including operators such as AND, OR, NOT, truncations, and phrase searches.
	\item[Search tool:] Library catalogs, websites, search engines, subject databases, journal portals searched, as well as notes on manual searches, expert inquiries, etc.
	\item[Selection criteria:] Criteria that determined whether a document was kept, for example:
		\begin{itemize}
			\item relevance to the research question
			\item degree of substantive differentiation
			\item verifiability of contents
			\item currency
			\item recognisability and reputation of the author
			\item reputation (reliability, expertise) of the source
			\item intended audience (specialist readership)
		\end{itemize}
	\item[Retained document:] Bibliographic information for the selected document.
	\item[Source type:] Journal article, book chapter, monograph, dissertation, newspaper article, encyclopedia entry, website article, research project report, etc.
	\item[Content:] Main contents of the document. Indicate whether it primarily contains:
		\begin{itemize}
			\item descriptive knowledge (definitions, empirical facts)
			\item explanatory knowledge (theories, statements about relationships)
			\item evaluative knowledge (statements about values and norms)
			\item procedural knowledge (methods, instruments, techniques)
		\end{itemize}
	\item[Relevance:] Mark the importance using the following symbols:
		\begin{itemize}
			\item $***$ = very important
			\item $**$ = important
			\item $*$ = of limited importance
		\end{itemize}
\end{description}

\section{Preparatory Research}

The general idea is to create an early AI System that detects suicidal humans by analysing their social media use.

But the new EU AI Act aims to regulate the use of artificial intelligence (AI) to prompte trustworthy AI, while ensuring protection against harmful effects of AI systems. (https://artificialintelligenceact.eu/article/1/) An suicide detection AI is bound to this Act as an High-Risk AI System and therefore needs to be auditable, which is still a difficult task, when using a complex model such as an LLM.

One of the options, although not yet researched well on llms is the use of SHAP Values. (https://christophm.github.io/interpretable-ml-book/shapley.html)



\section{Main Question 1}
Question: What are indications in a text, that someone is suicidal?

\subsection{Search Strategy}
\begin{itemize}
    \item \textbf{Date:} 06.01.2026
.   \item \textbf{Duration:} 30 min
    \item \textbf{Search terms:} indication suicide text
    \item \textbf{Search tool:} google schoolar
\end{itemize}

\subsection{Selection Criteria}

\begin{itemize}
	\item relevance to Main Question 1
	\item actuallity
	\item reputation
\end{itemize}

\begin{table}[H]
    \centering
    \begin{tabular}{|p{5cm}|p{2.5cm}|p{5cm}|p{2cm}|}
        \hline
        Retained document & Source type & Content & Relevance \\
        \hline
        Identification of Imminent Suicide Risk Among YoungAdults using Text Messages & proceeding & descriptive knowledge & *** \\
        \hline
		Words of Suicide: Identifying Suicidal Risk in Written Communications & inproceeding & descriptive knowledge & *** \\
        \hline
		Predicting the Risk of Suicide by Analyzing the Text of Clinical Notes & inproceeding & evaluative knowledge & ** \\
        \hline
		Online suicide prevention through optimised text classification & article & evaluative knowledge & ** \\
        \hline
		Analyzing Suicide Risk From Linguistic Features in Social Media: Evaluation Study & article & evaluative knowledge & * \\
        \hline
		The Youth Risk Behavior Surveillance System: Updating Policy and Program Applications & article & descriptive knowledge & * \\
        \hline
    \end{tabular}
    \caption{Research log for Main Question 1}
    \label{tab:research_log_mq1}
\end{table}

\section{Main Question 2}
Question: Can Shapley Values be used to explain how an LLM acts?

\subsection{Search Strategy}
\begin{itemize}
    \item \textbf{Date:} 06.01.2026
.   \item \textbf{Duration:} 30 min
    \item \textbf{Search terms:} SHAPley LLM xAI
    \item \textbf{Search tool:} google schoolar
\end{itemize}

\subsection{Selection Criteria}

\begin{itemize}
	\item relevance to Main Question 2
	\item actuallity
	\item reputation
\end{itemize}

\begin{table}[H]
    \centering
    \begin{tabular}{|p{5cm}|p{2.5cm}|p{5cm}|p{2cm}|}
        \hline
        Retained document & Source type & Content & Relevance \\
        \hline
		\hline
        Explainable artificial intelligence (XAI): from inherent explainability to large language models & misc & evaluative knowledge & *** \\
        \hline
		Explaining Large Language Models Decisions Using Shapley Values & misc & evaluative knowledge & * \\
        \hline
		Explainable AI Frameworks for Large Language Models in High-Stakes Decision-Making & inproceedings & evaluative knowledge & ** \\
        \hline
		llmSHAP: A Principled Approach to LLM Explainability & misc & descriptive knowledge & *** \\
        \hline
        
    \end{tabular}
    \caption{Research log for Main Question 2}
    \label{tab:research_log_mq2}
\end{table}

\end{document}
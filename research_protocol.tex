% !TEX encoding = UTF-8 Unicode
% !!!  THIS FILE IS UTF-8 !!!
% !!!  MAKE SURE YOUR LaTeX Editor IS CONFIGURED TO USE UTF-8 !!!
% ---------------------------------------------------------------
% Author: Adrian Joost
% ---------------------------------------------------------------

%-------------------------
% header
% ------------------------
\documentclass[a4paper,12pt]{scrartcl}
\linespread {1.25}

%-------------------------
% packages and config
% ------------------------
\input{packages_and_configuration}

%-------------------------
% document begin
%-------------------------
\begin{document}

\title{Research protocol}

\section{Introduction}

This research protocol describes how to record the search for information sources for a bachelor thesis. For each of the usually two to four main questions presented in the thesis introduction, a log of the search for sources is kept in parallel with the research process. Concise records are also kept of search attempts that were not very successful and did not yield suitable documents. The preliminary research that led to the formulation of the research question should be documented under "Preparatory Research"; this may be given as a summary rather than a detailed table.

The template provided here for the research protocol can be adapted continuously (for example as a Word file) to fit the individual search process. The completed protocol may be appended to the thesis as an annex without the instructions and example included here.

The columns of the research protocol are to be understood as follows:
\begin{description}
	\item[Date:] Date of the search.
	\item[Duration:] Time spent on the individual search (in minutes).
	\item[Search terms:] Words and combinations of words used in the search, including operators such as AND, OR, NOT, truncations, and phrase searches.
	\item[Search tool:] Library catalogs, websites, search engines, subject databases, journal portals searched, as well as notes on manual searches, expert inquiries, etc.
	\item[Selection criteria:] Criteria that determined whether a document was kept, for example:
		\begin{itemize}
			\item relevance to the research question
			\item degree of substantive differentiation
			\item verifiability of contents
			\item currency
			\item recognisability and reputation of the author
			\item reputation (reliability, expertise) of the source
			\item intended audience (specialist readership)
		\end{itemize}
	\item[Retained document:] Bibliographic information for the selected document.
	\item[Source type:] Journal article, book chapter, monograph, dissertation, newspaper article, encyclopedia entry, website article, research project report, etc.
	\item[Content:] Main contents of the document. Indicate whether it primarily contains:
		\begin{itemize}
			\item descriptive knowledge (definitions, empirical facts)
			\item explanatory knowledge (theories, statements about relationships)
			\item evaluative knowledge (statements about values and norms)
			\item procedural knowledge (methods, instruments, techniques)
		\end{itemize}
	\item[Relevance:] Mark the importance using the following symbols:
		\begin{itemize}
			\item $***$ = very important
			\item $**$ = important
			\item $*$ = of limited importance
		\end{itemize}
\end{description}

\section{Preparatory Research}

\section{Main Question 1}
Question: How did X affect Y?

\subsection{Search Strategy}
\begin{itemize}
    \item \textbf{Date:} 01.01.2024
    \item \textbf{Duration:} 30 min
    \item \textbf{Search terms:} example search terms
    \item \textbf{Search tool:} example search tool
\end{itemize}

\subsection{Selection Criteria}
List inclusion/exclusion criteria clearly in bullet points.
\begin{table}[H]
    \centering
    \begin{tabular}{|p{5cm}|p{2cm}|p{5cm}|p{2cm}|}
        \hline
        Retained document & Source type & Content & Relevance \\
        \hline
        example retained document & example source type & example content & *** \\
        \hline
    \end{tabular}
    \caption{Research log for Main Question 1}
    \label{tab:research_log_mq1}
\end{table}

\section{Main Question 2}

\end{document}